\section{\fluent{}}\seclabel{Fluent}
In this section, we present \fluent{}: a prototype distributed debugging
framework that leverages the theoretical foundations of \watprovenance{} and
\watprovenance{} specifications.

\fluent{} uses \watprovenance{} specifications to generate the provenance of
data as it transits through the black box components of a distributed system.
To write a \watprovenance{} specification for a black box, a developer must
first wrap the black box in a \fluent{} shim. A shim acts as proxy,
intercepting all inbound requests sent to a black box and all outbound replies
produced by a black box. \fluent{} shims provide two key pieces of
functionality.

First, \fluent{} shims are responsible for recording the trace $T$ of requests
that are sent to a black box, as well as the corresponding replies produced by
the black box. These traces are later used as the inputs to \watprovenance{}
specifications.  Currently, \fluent{} shims persist traces in a relational
database.

Second, \fluent{} shims implement a simple distributed tracing service.
Whenever a \fluent{} shim receives a request, it records the address of the
message's sender along with the request. Similarly, whenever a \fluent{}
shim sends a request, it records the address of the message's
destination.
%
This enables a developer to integrate the \watprovenance{} of multiple black
boxes within a distributed system. To find the cause of a particular black box
output, we invoke the black box's \watprovenance{} specification.  The
specification returns the set of witnesses that cause the output. Then, we can
trace a request in a witness back to the black box that sent it and repeat the
process, invoking the sender's \watprovenance{} specification to get a new set
of witnesses.

After a user has written a black box's shim, they can write the black box's
\watprovenance{} specification. \fluent{} \watprovenance{} specifications are
simple scripts written in a developer's choice of either SQL or Python. Given a
particular black box request, a \watprovenance{}  script computes the
corresponding \watprovenance{} with respect to the black box's trace (which is
persisted in a relational database by the black box's shim).
