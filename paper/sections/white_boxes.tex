\newcommand{\dedalusplus}{Dedalus$^+$}

\section{White Boxes}\seclabel{WhiteBoxes}
Recall from \secref{BlackBoxes} that real-world distributed systems are
composed of commonly used open source services (i.e.\ black boxes) and
application-specific business logic that is written from scratch (i.e.\ white
boxes).  In \secref{BlackBoxes}, we saw that automatically inferring the
\watprovenance{} of a black box is intractable. Is the same true for white
boxes? Not quite. If we implement a white box from scratch, we have the
opportunity to implement it in a programming language for which we \emph{can}
automatically extract \watprovenance{}. In this section, we propose one such
programming language. We also outline the drawbacks and limitations of the
white box approach to computing \watprovenance{} and discuss alternatives.

\subsection{\dedalusplus}
Dedalus is a variant of Datalog that introduces notions of time and mutable
state~\cite{alvaro2011dedalus, marczak2012confluence, alvaro2013declarative}.
Bloom is a concrete implementation of Dedalus that has been used to program
a variety of distributed systems, including key-value stores, state machine
replication protocols, and distributed shopping
carts.~\cite{alvaro2011consistency, conway2012logic}. In this section, we
describe how we can automatically extract the \watprovenance{} of a program
written in \dedalusplus: the variant of Dedalus without negation.

A \dedalusplus{} state machine stores its state in a collection of typed
relations, much like a relational database. Upon receiving a request, a
\dedalusplus{} state machine executes a positive Datalog program to (a) update
its state and (b) compute a response for the request. For example, consider the
simple \dedalusplus{} Twitter clone shown in \figref{DedalusTwitter}. The
program stores its state in two relations: (1) \texttt{Users} which records the
Twitter handle and name of all users and (2) \texttt{Following} which records
the handles that a particular user is following. The program supports three
types of requests:
\begin{itemize}
  \item
    \texttt{register\_user}
\end{itemize}

TODO:
  - briefly describe the Twitter clone
  - show a provenance graph for a particular trace
  - describe how we can extract the wat provenance

{\begin{figure}[ht]
  \centering
  {
    \small
    \begin{Python}[gobble=3]
      Users(handle: str, name: str)
      Following(follower: str, followee: str)

      def register_user(handle: str, name: str):
        Users += {(handle, name)}
        return {("ok")}

      def follow(me: str, them: str):
        Following += {(me, them)}
        return {("ok")}

      def following(me: str):
        return $\pi_{\texttt{name}}$($\sigma_{\texttt{follower=me}}$(Following) $\join_{\text{handle=followee}}$ Users)
    \end{Python}
  }
  \caption{\dedalusplus{} Twitter}
  \figlabel{DedalusTwitter}
\end{figure}
}

\subsection{Limitations and Alternatives}
TODO:
  - Admit that \dedalusplus{} is not the most expressive language in the world.
    We can't really delete anything.
  - Say that adding in negation to get the full power of Dedalus makes the
    language to complication to get wat provenance out of in a nice way.
    Negative provenance is not super well understood.
  - Say that there are alternatives to wat provenance that are well suited to
    more expressive languages like Dedalus. Mention network provenance as an
    example.
  - Say that even if you write a program in a normal programming language, it's
    probably still ok. You can use normal debugging tools on it to walk through
    what's going on. It's ad-hoc but if the program is simple, it's probably
    ok.  Contrast this with black boxes, where this is very onerous because you
    didn't write the code.
