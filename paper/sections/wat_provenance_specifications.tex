\section{\WatProvenance{} Specifications}\seclabel{WatProvSpecs}
Now that we've defined \watprovenance{}, we turn to the matter of computing it.
Real-world distributed systems are typically a composition of two types of
components: (1) application specific services that are written from scratch
(i.e.\ ``business logic''), dubbed \defword{white boxes}, and (2) widely used
open source services that are not written by the application developer, dubbed
\defword{black boxes}, (e.g., Redis, Postgres). In this section, we discuss how
to compute the \watprovenance{} of black boxes. In \secref{WhiteBoxes}, we
discuss white box \watprovenance{}.

\subsection{Provenance Specifications}
Automatically computing the \watprovenance{} for an \emph{arbitrary} black box
is more or less hopeless. In order to do so, we would have to analyze an
arbitrary piece of code written in an arbitrary programming language and infer
an abstract state machine that accurately models the black box. Then, to
compute the \watprovenance{} of a particular request with respect to a given
trace, we would have to perform some sort of complex code analysis over a large
body of code. Worse, we may not even have access to the source code of the
black box!  For example, we don't have access to the source code of cloud
services like Amazon S3 or Google Cloud Spanner.

\newcommand{\kvget}{\texttt{get}}
\newcommand{\kvset}{\texttt{set}}

Though automatically computing the \watprovenance{} for an \emph{arbitrary}
black is intractable, we can take advantage of the fact that most real-world
black boxes are far from arbitrary. Many black boxes have complex
implementations but are designed with very simple interfaces.  This allows us
to sidestep the issue of \emph{inferring} \watprovenance{} from an
implementation and instead \emph{specify} \watprovenance{} directly from an
interface. That is, we can write a \defword{\watprovenance{} specification}: a
function that---given a trace $T$ and request $i$---directly returns the
\watprovenance{} $\Wat{M, T, i}$ for a black box modeled as state machine $M$.

For example, if we restrict our attention to the \kvget{} and \kvset{} API of
Redis (as with \exampleref{WatExampleXyx}), then the \watprovenance{}
specification of a \kvget{} request is trivial: \textbf{the \watprovenance{} of
a \kvget{} request for key $k$ includes only the most recent \kvset{} to
$k$}. Redis is implemented in over 50,000 lines of C, so automatically
inferring the \watprovenance{} of a \kvget{} request is intractable. This
\watprovenance{} specification avoids the intractability and instead specifies
the \watprovenance{} in a single line of text.

Moreover, codifying this one line \watprovenance{} specification is expectedly
straightforward. The programming language in which we write \watprovenance{}
specifications is unimportant. In \figref{RedisProvSpec}, we provide a simple
implementation of the one line \watprovenance{} specification in Python. The
specification, \texttt{get\_prov}, takes in a trace \texttt{T} and a \kvget{}
request \texttt{i} for key $k$. Redis requests are represented as objects of
type \texttt{Request} with subclasses \texttt{GetRequest} and
\texttt{SetRequest}.  \texttt{get\_prov} iterates through the trace in reverse
order, looking for a \kvset{} request to key $k$. If such a \kvset{} request is
found, \texttt{get\_prov} returns it. Otherwise, \texttt{get\_prov} returns an
empty witness.
% Recall that the \watprovenance{} of a request is a set of witnesses closed
% under supertrace in $T$, which is why \texttt{get\_prov} returns either a set
% of the singleton witness \texttt{[a]} or a set of the empty witness
% \texttt{[]}.

{\begin{figure}[ht]
  \centering
  \inputpython{figures/redis_prov_spec.py}
  \caption{Redis \kvget{} request \watprovenance{} specification}
  \figlabel{RedisProvSpec}
\end{figure}
}

In \secref{Fluent}, we present a prototype implementation of a system for
writing \watprovenance{} specifications and describe the details of concretely
how traces are collected and how \watprovenance{} specifications are written.
In this section, we omit the details and focus on the concepts behind
\watprovenance{} specifications.

\subsection{Examples}
The \watprovenance{} specification of a Redis \kvget{} request is particularly
simple, but this simplicity is the rule, not the exception. We now survey a
variety of popular black boxes and demonstrate that their \watprovenance{}
specifications are mostly very simple.

\paragraph{Stateless Services}
A \defword{stateless service} is a service that can be modelled as a state
machine with a single state. Given a request, a stateless service always
produces the same reply, no matter what other requests it has already serviced.
For example, a web server serving a static website is stateless; it replies to
all requests with the same website. Similarly, cloud services like Google Cloud
Vision API and Google Cloud Speech-to-Text are also stateless. Given an image
or audio clip, these services deterministically return a description of the
image or a transcription of the audio. \Watprovenance{} specifications of a
stateless service are trivial. Requests are completely independent, so the
\watprovenance{} of any request consists only of the empty witness.

\paragraph{Key-Value Stores}
We've already seen a \watprovenance{} specification for the \kvget{} and
\kvset{} API of Redis. This specification works equally well for any other
key-value store like Voldemort, Riak, or Memached. Moreover, we can easily
extend our \watprovenance{} specification to handle more of Redis' API. For
example, consider the operations \texttt{append}, \texttt{decr},
\texttt{decrby}, \texttt{incr}, and \texttt{incrby} which all modify the value
associated with a particular key. With these operations present in a trace, the
\watprovenance{} specification for a \kvget{} request to key $k$ now includes
the most recent set to $k$ and all subsequent modifying operations to key $k$.

\paragraph{Object Stores}
We can also write \watprovenance{} specifications for storage systems that are
more complex than key-value stores. For example, consider an object store like
Amazon S3 where users can create, move, copy, list, and get buckets and
objects. The \watprovenance{} specification for the get of an object $o$ in
bucket $b$ includes the most recent creation of the bucket $b$ (either by
creation or moving) and the most recent creation of $o$ (again, either by
creation or moving). The \watprovenance{} of a request to list the contents of
a bucket includes the most recent creation of the bucket, the most recent
creation of every object in the bucket, and the deletion of any object that was
previously in the bucket.

\paragraph{Distributed File Systems}
We can also specify the \watprovenance{} of a distributed file system like NFS
or HDFS. A \watprovenance{} specification of a request to read a byte range
from a file includes the most recent creation of the file, the most recent
creation of the parent directories of the file, and the most recent writes that
overlap with the requested byte range.

\paragraph{Coordination Services}
Systems use coordination services like Apache
Zookeeper~\cite{hunt2010zookeeper} and Chubby~\cite{burrows2006chubby} for
leader election, mutual exclusion, etc. Take Zookeeper as an example.
Zookeeper's API resembles that a of a file system; users can create, delete,
write, and read file-like objects called znodes. Though the implementation of
Zookeeper and HDFS are radically different, their APIs (and thus their
\watprovenance{} specifications) are similar. For example, the \watprovenance{}
specification of a request to read a znode includes the most recent creation of
the znode and the most recent creations of all ancestor znodes.

\paragraph{Load Balancers}
Consider a load balancer, like HAProxy, that is balancing load across a set
$s_1, \ldots, s_n$ of $n$ servers. Periodically, a server $s_i$ sends a
heartbeat to the load balancer that includes $s_i$'s average load for the last
five minutes. Whenever the load balancer receives a message from a client, it
forwards the message to the server $s_i$ that is least loaded. Modelling the
forwarding decision $s_i$ as the output of the load balancer, the
\watprovenance{} specification for the forwarding decision includes the most
recent heartbeat message from the least loaded server.

\subsection{Discussion}
These examples show that many popular black boxes have simple \watprovenance{}
specifications and that many \watprovenance{} specifications are similar to one
another. We posit that this is not a coincidence. System designers actively
design their systems to be as simple and as familiar as possible. If a black
box had a very complex API with very complex dependencies between requests,
then it would be challenging for a user to become proficient in using the
system. Take Zookeeper as an example. The implementation and use cases of
Zookeeper are radically different than that of a traditional distributed file
system like HDFS. However, by mimicking the API of a file system, Zookeeper's
API is immediately familiar to developers. As a result, Zookeeper's
\watprovenance{} specification is very similar to HDFS's \watprovenance{}
specification, both of which are relatively simple.

However, not \emph{every} black box has a simple provenance specification. Take
for example, a relational database where users can insert tuples, remove
tuples, and issue queries (as with \exampleref{WatExampleSetDiff}). As
\thmref{WatSubsumesWhy} shows, writing a \watprovenance{} specification for a
particular query is tantamount to specifying the \whyprovenance{} of the query.
Worse, if the query includes constructs like set difference or aggregation that
are beyond the scope of \whyprovenance{}, the \watprovenance{} specification
becomes even more challenging to write. Fortunately, the effort of writing a
\watprovenance{} specification need not be duplicated. Black boxes like Redis
are written by a handful of developers but used by thousands of others.
Provenance specifications are no different. Once a single developer writes a
provenance specification for a particular black box, all other users of the
black box can leverage it.

Furthermore, there is a simple yet effective way to unit test provenance
specifications that we call \defword{exhaustive \watprovenance{} checking}.
Given a state machine $M$ with start state $s_0$, a trace $T$, and an input
$i$, we can automatically compute the \watprovenance{} of $i$ by exhaustively
computing $\epsilon^*(s_0, T'i)$ for every subtrace $T'$ of $T$.  Doing so, we
can automatically compute the \watprovenance{} of a particular request and
verify that it matches the \watprovenance{} produced by a provenance
specification.  Unfortunately, this technique runs in superexponential time
with respect to the length of $T$, so it is effective for small unit tests but
impractical for computing the \watprovenance{} of real-world traces.

\subsection{Limitations}
Automatically inferring the \watprovenance{} of a black box is intractable, and
while \watprovenance{} specifications work well in the common case, they are
not a panacea. Here, we discuss two limitations of \watprovenance{}
specifications.

\paragraph{Buggy Black Boxes}
We have thus far tacitly assumed that common black boxes like Redis faithfully
implement their advertised interfaces. However, if a black box is buggy and
deviates from its expected behavior, then a \watprovenance{} specification will
likely produce erroneous provenance. Fortunately, black box bugs are relatively
rare (at least compared to the frequency of bugs in application-specific code
written from scratch). Moreover, we can use exhaustive \watprovenance{}
checking to identify potential bugs in a black box. If the \watprovenance{}
computed by an exhaustive \watprovenance{} check differs from the
\watprovenance{} returned by a specification, it's an indication that the black
box might be deviating from its intended specification.

\paragraph{Nondeterministic Black Boxes}
The definition of \watprovenance{} assumes a \emph{deterministic} state
machine, yet some black boxes---e.g., a load balancer that forwards requests
uniformly at random---are nondeterministic. We leave a generalization of
\watprovenance{} and \watprovenance{} specifications to nondeterministic state
machines as an interesting avenue for future work.
