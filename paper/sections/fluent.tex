\section{\fluent{}}\seclabel{Fluent}
In this section, we present \fluent{}: a prototype distributed debugging
framework that leverages the theoretical foundations of \watprovenance{} and
\watprovenance{} specifications. We also describe how to use \fluent{} to debug
our motivating example from \secref{WatProvenanceExample}.

\subsection{\WatProvenance{} Specifications}
{\begin{figure}[ht]
  \centering
  \tikzstyle{minw}=[minimum width=1cm]
  \tikzstyle{minh}=[minimum height=1cm]
  \tikzstyle{wb}=[align=center, draw=black, fill=white, text=black]
  \tikzstyle{bb}=[align=center, draw=black, fill=black, text=white]

  \begin{tikzpicture}
    \node[wb, minw, minh] (wb) at (0, 0) {Client};
    \node[wb, minw, rotate=90] (shim) at (2.5, 0) {Shim};
    \node[bb, minw, minh] (bb) at (4, 0) {Black\\Box};

    \tikzstyle{e}=[thick, -latex]
    \tikzstyle{n}=[inner sep=0, semithick]
    \draw[e] ($(wb.north east)!0.25!(wb.south east)$)
             to node[n, above=0.1cm] {\relsize{-1}1}
             ($(shim.north west)!0.75!(shim.north east)$);
    \draw[e] ($(shim.south west)!0.75!(shim.south east)$)
             to node[n, above=0.1cm] {\relsize{-1}2}
             ($(bb.north west)!0.25!(bb.south west)$);
    \draw[e] ($(bb.north west)!0.75!(bb.south west)$)
             to node[n, below=0.1cm] {\relsize{-1}3}
             ($(shim.south west)!0.25!(shim.south east)$);
    \draw[e] ($(shim.north west)!0.25!(shim.north east)$)
             to node[n, below=0.1cm] {\relsize{-1}4}
             ($(wb.north east)!0.75!(wb.south east)$);
  \end{tikzpicture}
  \caption{A \fluent{} shim}
  \figlabel{FluentShim}
\end{figure}
}

\fluent{} uses \watprovenance{} specifications to generate the provenance of
data as it transits through the black box components of a distributed system.
In order to write a \watprovenance{} specification for a black box, a developer
must first wrap the black box in a \fluent{} shim, as illustrated in
\figref{FluentShim}. A shim acts as proxy, intercepting all inbound requests
sent to a black box and all outbound replies produced by a black box. \fluent{}
shims provide two key pieces of functionality:
\begin{enumerate}
  \item
    \fluent{} shims are responsible for recording the trace $T$ of requests
    that are sent to a black box, as well as the corresponding replies produced
    by the black box. These traces are later used as the inputs to
    \watprovenance{} specifications.  Currently, \fluent{} shims persist traces
    in a relational database.

  \item
    \fluent{} shims implement a simple distributed tracing service. Whenever a
    \fluent{} shim receives a request, it records the address of the message's
    \emph{sender} along with the request. Similarly, whenever a \fluent{} shim
    sends a request, it records the address of the message's \emph{destination}.
    %
    This enables a developer to integrate the \watprovenance{} of multiple
    black boxes within a distributed system. To find the cause of a particular
    black box output, we invoke the black box's \watprovenance{} specification.
    The specification returns the set of witnesses that cause the output. Then,
    we can trace a request in a witness back to the black box that sent it and
    repeat the process, invoking the sender's \watprovenance{} specification to
    get a new set of witnesses.
\end{enumerate}

After a user has written a black box's shim, they can write the black box's
\watprovenance{} specification. \fluent{} \watprovenance{} specifications are
simple scripts written in either SQL or Python. Given a particular black box
request, a \watprovenance{}  script computes the corresponding \watprovenance{}
with respect to the black box's trace (which is persisted in a relational
database by the black box's shim). For example, a SQL \watprovenance{}
specification for Redis get requests is listed in
\figref{ConcreteRedisProvSpec}. The \watprovenance{} specification consists of
a single SQL query parameterized by the request's key \texttt{KEY} and logical
time \texttt{TIME}. As with \figref{RedisProvSpec}, this \watprovenance{}
specification returns the most recent set to the key \texttt{KEY}. The
\texttt{redis\_set\_req} relation is a view over the trace that only includes
set requests.

\begin{figure}[ht]
  \begin{Verbatim}
    SELECT *
    FROM redis_set_req
    WHERE key = $KEY AND time <= $TIME
    ORDER BY time DESC
    LIMIT 1
  \end{Verbatim}
  \caption{\Watprovenance{} specification for Redis get requests}
  \figlabel{ConcreteRedisProvSpec}
\end{figure}

\subsection{Our Motivating Example}
\newcommand{\systemname}{ZardozBook}
We now return to the motivating example from \secref{WatProvenanceExample} and
explore how \fluent{} enables us to diagnose why Bob was able to see Ava's mean
comment. We assume that we have already written \fluent{} shims and
\watprovenance{} specifications all the components in \figref{MeanComment}.

\begin{itemize}
  \item
    We begin by inspecting the trace of Bob's \systemname{} client. The trace
    includes a message with Ava's mean comment. Inspecting the source address
    stored alongside the message, we find that the message was sent by the load
    balancer.
  \item
    We then examine the load balancer's trace. The load balancer's
    \watprovenance{} specification informs us that the load balancer sent Ava's
    mean comment to Bob's \systemname{} client as a response to receiving the
    mean comment from application server $s_3$.
  \item
    We then look at the application server $s_3$'s trace. The application
    server sent the mean comment to the load balancer as a reply to a
    \texttt{GET} request that was previously sent by Bob's \systemname{}
    client. The application server's \watprovenance{} specification informs us
    that the application server produced the mean comment because it previously
    received it from the centralized Postgres database.
  \item
    We then inspect the Postgres database's trace. We find that the centralized
    database sent the mean comment to the application server as a reply to a
    cache refresh request that was previously sent to the Postgres database by
    application server $s_3$. The \watprovenance{} of the Postgres database
    reveals that the Postgres database produced the mean comment because it was
    previously inserted into the database by application server $s_3$.
  \item
    We then review application server $s_2$'s trace. $s_2$'s \watprovenance{}
    specification shows that $s_2$ sent the mean comment to the Postgres
    database because it previously received the mean comment from the load
    balancer.
  \item
    We again examine the load balancer's trace and follow the mean comment back
    to Ava's \systemname{} trace.
  \item
    Ava's trace shows that Ava issued a request to unfriend Bob before she
    posted the mean comment. Following the trace of the unfriend request, we
    find that it was forwarded to application server $s_1$ by the load
    balancer and that application server $s_1$ later propagated the request to
    Postgres.
  \item
    Examining the Postgres trace, we find that it received the unfriend request
    \emph{after} the mean comment, even though Ava issued the unfriend request
    \emph{before} the mean comment.
\end{itemize}

From this analysis, we discover that because the load balancer can forward
requests from a single client to different application servers and because
application servers lazily synchronize with the centralized Postgres
repository, client requests can be effectively reordered. Using this insight, a
\systemname{} engineer can enable sticky sessions in the load balancer,
ensuring that the load balancer forwards requests within a single session to a
single application server. This fixes the bug and prevents Bob from
experiencing any future race condition woes.
