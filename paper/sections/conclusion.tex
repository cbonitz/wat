\section{On White Boxes}\seclabel{Discussion}
In \secref{WatProvSpecs}, we saw how easy it is to write shims for a variety of
common black box components. In return for this modest investment, \fluent{}
allows programmers to ask and answer debugging questions that span node
boundaries and involve mutable state that changes over time. It is tempting to
ask if we can take the idea further. If our components are written in a
language amenable to provenance collection---if we work only with ``white boxes''---can we infer their \watprovenance{}
specifications automatically? Or, better still, can we obtain a form of
provenance even richer than \watprovenance{}: one that explains not just
\emph{which} input tuples contribute to an output tuple, but also precisely
\emph{how}?

Unfortunately, automatically extracting \watprovenance{} remains elusive, even
for state machines written in a restricted white-box language for provenance.
Take for example provenance-enhanced distributed logic programming languages
such as NDLog~\cite{loo2006design} and Dedalus~\cite{alvaro2011dedalus} that
store system state in relations and represent programs as collections of
relational queries. Each exposes a form of \whyprovenance{} that accounts for
state that changes over time. In an NDLog program, effect-producing
events---such as the addition, deletion, or update of records---are reified
into the program's provenance graph. In Dedalus, logical time is reified into
all of the tuples.
%
% \jmh{%
%   Peter -- can you reword the notions of reification here? In Dedalus it's
%   ``recorded as an attribute in every tuple''. In NDlog/DistTape, what is it
%   precisely -- events are recorded as data tuples added to the system state?
% }
%
These custom representations of provenance in time, while not formalized, are
similar in spirit to \watprovenance{}. It turns out, however, that they are
\emph{not} equivalent to \watprovenance{}---neither Dedalus nor NDLog
provenance necessarily identify minimal witnesses. The underlying issue is the
problem of \defword{negative provenance} or the provenance of
non-answers~\cite{chapman2009whynot,huang2008nonanswers}, which remains an open
issue in the database research community.

\newcommand{\kvset}{\text{set}}
\newcommand{\kvget}{\text{get}}
\newcommand{\freeze}{\text{freeze}}
\newcommand{\trunc}{\text{trunc}}

Consider the key-value state machine presented in \exampleref{WatExampleKvs},
extended to support two new requests. The freeze request makes the existing set of keys and
values immutable, while trunc deletes all of the keys and values (provided that
freeze has not already been called). Now consider the trace
\[
  T = a_1 a_2 a_3 = \kvset(x,1); \freeze; \trunc
\]
The \watprovenance{} of $i = \kvget(x)$ is the singleton set $\set{a_1a_2}$
consisting of the single witness $a_1a_2$. But explaining why $a_2$ is part of
this witness requires reasoning about negative provenance: $x$ had the value $1$ both
because of $a_1$ (which inserted the value into the store) and because $a_3$
had \emph{no effect} (due to the action of $a_2$, \emph{without which} $a_3$ would have
removed the effects of $a_1$!). If the state machine was written in a language
such as NDLog or Dedalus, implementing it would require the use of logical
negation to capture the reasoning that trunc applies only if freeze \emph{does not exist} before it.
 (In fact, \exampleref{WatExampleXyx} through
\exampleref{WatExampleSetDiff} all require nonmonotonic logic.) Explanations of the output to $i$ would therefore require both positive and negative
provenance. Explaining why a particular event \emph{did not occur} is
intractable in general, as the explanation may be infinitely large.  Existing
techniques for collecting negative provenance apply heuristics that
over-approximate the why provenance of non-answers. For example, Dedalus
provenance would determine that the provenance of $i$ is $\set{a_1a_2a_3}$.

\newcommand{\dedalusplus}{Dedalus$^+$}
We could, of course, constrain the white-box programming language to rule out the
complication of negative provenance.  The language
\dedalusplus~\cite{marczak2012confluence} is the positive fragment of Dedalus,
in which negation is not permitted (except on base relations). Programs written
in \dedalusplus{} generate positive \whyprovenance{} graphs whose leaves
correspond exactly with the \watprovenance{} of the given execution.
Unfortunately, \dedalusplus{} is not adequately expressive to implement
arbitrary stateful distributed services. The CALM
theorem~\cite{alvaro2011consistency,ameloot2013transducers} shows that
programs produce a single consistent output under all input orders
if and only if those programs do not use negation in their logic. Therefore
programs whose outputs are (by definition) dependent upon their input
orders---such as mutable key/value stores, file systems and object stores,
caches, etc---cannot be implemented in a logic language \emph{without} using
negation! Negative provenance is requirement for capturing the semantics of
typical stateful services.

In short, fully automatic collection of \watprovenance{} for general-purpose
systems seems tricky. As future work, we plan to continue efforts to design of
an expressive programming language for which we can automatically extract
\watprovenance{} with a minimum of programmer assistance.

\section{Conclusion}\seclabel{Conclusion}
This paper identified inadequacies in existing formalisms that are used to
reason about the causes of events in distributed systems. Causality
overapproximates the true cause of an event, and data provenance is restricted
to the domain of static relational databases.  We then presented
\watprovenance{}: a novel form of provenance that generalizes \whyprovenance{}
and refines causality. \Watprovenance{} formalizes precisely why a particular
state machine produces a particular output with respect to a previously
executed trace of inputs. We then discussed how to sidestep the complexity of
automatic \watprovenance{} extraction with \watprovenance{} specifications,
which we found to be simple to write in practice due to the fact that most
distributed systems components have simple interfaces. We implemented our
theoretical findings in a system called \fluent{} and found it to produce
lineage that is orders of magnitude more succinct than existing tools targeted
at debugging distributed systems.

\begin{acks}
  The authors would like to thank Lennart Oldenburg, Sanjay Krishnan, Alvin
  Cheung, and Anthony Tan for fruitful discussion and feedback.
  %
  This research is supported in part by DHS Award HSHQDC-16-3-00083, NSF CISE
  Expeditions Award CCF-1139158, and gifts from Alibaba, Amazon Web Services, Ant
  Financial, CapitalOne, Ericsson, GE, Google, Huawei, Intel, IBM, Microsoft,
  Scotiabank, Splunk and VMware.
\end{acks}
