\begin{abstract}
  There is currently no systematic way to reason about the fine-grained causes
  of events in a real-world distributed system. Causality, from the distributed
  systems literature, can be used to compute the causal history of an arbitrary
  event in a distributed system, but the causal history of an event is a vast
  overapproximation of the true causes. Data provenance, from the database
  literature, describes precisely the cause of a particular output of a
  relational query, but data provenance is limited to the domain of static
  relational databases. In this paper, we present \defword{\watprovenance{}}: a
  novel form of provenance that unifies causality and data provenance. Given an
  arbitrary time-varying state machine, \watprovenance{} describes why the
  state machine produces a particular output when given a particular input.
  This enables system developers to systematically reason about the causes of
  events in real-world distributed systems. We find that automatically
  extracting the \watprovenance{} of arbitrary state machines is often
  challenging and sometimes impossible due to their complex implementations.
  Fortunately, many distributed systems components have simple interfaces from
  which a developer can directly specify their \watprovenance{} using a
  technique we call \defword{\watprovenance{} specifications}. Using
  \watprovenance{} as a theoretical foundation, we implement a prototype
  distributed debugging framework called \fluent{} and empirically demonstrate
  that \watprovenance{} makes debugging distributed systems significantly
  easier than what is possible with existing techniques.
\end{abstract}
