\section{Introduction}\seclabel{Introduction}
Debugging distributed systems is hard. Traditional debugging techniques are
poorly suited to distributed systems in which bugs arise across multiple nodes
connected by an unreliable network that can drop, duplicate, and reorder
messages. Distributed debugging tools exist but are in their infancy. They help
tame \emph{some} of the complexities of distributed debugging, but have limited
applicability to real-world distributed systems that are made up of a large
number of complex components. Consequently, developers perform ad-hoc root
cause analysis to find the source of a bug, stitching together the logs of
multiple concurrently executing nodes.

Worse, existing formalisms are also inadequate to reason about systematically
debugging real-world distributed systems. For example, consider
\defword{causality}~\cite{lamport1978time}. Causality is a general-purpose
formalism that specifies the causal history of a particular event in an
arbitrary distributed system. However, causality is too general-purpose, as it
fails to incorporate any semantics of the underlying distributed
system~\cite{bailis2012potential}. As a consequence, the causal history of an
event is an overapproximation of the cause of the event. It includes all the
events that \emph{might} have caused a particular event instead of the events
that \emph{actually do} cause it.

Alternatively, consider data provenance in the form of
\defword{\whyprovenance{}}~\cite{cheney2009provenance, buneman2001and}. Given a
relational database, a query issued against the database, and a tuple in the
output of the query, \whyprovenance{} explains why the output tuple was
produced. That is, \whyprovenance{} produces the input tuples which, if passed
through the relational operators of the query, would produce the output tuple
in question. In contrast to causality, data provenance heavily incorporates the
semantics of relational databases and queries to describe \emph{precisely} the
cause of a particular output. However, \whyprovenance{} makes the critical
assumption that the underlying relational database is static. It cannot handle
the time-varying nature of stateful distributed systems. Moreover, data
provenance is limited to the domain of relational data and cannot easily be
applied to other system components (e.g., load balancers, file systems,
coordination services, etc.).

In short, causality lacks a notion of data dependence, and data provenance
lacks a notion of time. In this paper, we present \defword{\watprovenance{}}
(why-across-time provenance): a novel form of data provenance that unifies
ideas from the two. \Watprovenance{} generalizes \whyprovenance{} from the
domain of relational queries issued against a static database to the domain of
arbitrary time-varying state machines in a distributed system. More
specifically, given a deterministic state machine, the state machine's sequence
of inputs, and a particular input to the state machine, \watprovenance{}
formalizes \emph{why} the state machine produces the output that it does. This
description is the set of subsequences of the input trace that are necessary
and sufficient to generate the output in question. Borrowing from causality,
\watprovenance{} can be applied to time-varying state machines.  Borrowing from
\whyprovenance{}, \watprovenance{} incorporates state machine semantics to
avoid overapproximating provenance.

After we define \watprovenance{}, we turn to the matter of computing it. We
observe that automatically extracting the \watprovenance{} of a state machine
is often infeasible. Computing the \watprovenance{} of a state machine is
tantamount to inferring the state machine's data dependencies using a complex
code analysis of the state machine's source code. This source code can be both
large and complex which makes this code analysis intractable.
%
Though automatically extracting the \watprovenance{} of an \emph{arbitrary}
state machine is difficult, many distributed systems components are designed
with simple and minimalistic APIs. We can take advantage of this observation
and sidestep the complexity of \emph{extracting} the \watprovenance{} from the
\emph{implementation} of a state machine and instead \emph{specify} the
\watprovenance{} from the \emph{interface} of a state machine. To this end, we
propose \defword{\watprovenance{} specifications}: functions that directly
encode the \watprovenance{} of a state machine using its interface instead of
its implementation. We describe the provenance specifications of a number of
widely used distributed systems components (e.g.,\ Redis, Amazon S3, HDFS,
Zookeeper) and find that in practice, they are often straightforward to
implement.

Next, we present \fluent{}: a prototype distributed debugging framework that
leverages the formal foundations of \watprovenance{} and \watprovenance{}
specifications. \fluent{} includes a mechanism for developers to write
\watprovenance{} specifications that are executed against the input traces of
the components in a distributed system. We use \fluent{} to measure the
complexity of writing \watprovenance{} specifications and also compare
\fluent{} to existing distributed debugging frameworks.

This paper presents the following contributions:
\begin{itemize}
  \item
    We define \watprovenance{}: a formalism that extends notions of data
    provenance to the realm of state machines in a distributed system.
  \item
    We present \watprovenance{} specifications: a mechanism to compute the
    \watprovenance{} of distributed system components. We also describe a set
    of \watprovenance{} specifications for a number of widely used distributed
    systems components, illustrating that \watprovenance{} specifications can
    be straightforward to write.
  \item
    We implement a prototype distributed debugger called \fluent{} that
    leverages the theoretical foundations of \watprovenance{}. We demonstrate
    that \fluent{} makes it easier to identify the precise causes of events in
    a distributed system compared to existing debugging techniques.
\end{itemize}
