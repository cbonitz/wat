\newcommand{\ttt}[1]{\texttt{#1}}

\section{Background}\seclabel{Background}

\subsection{Causality}
As described by Lamport in \cite{lamport1978time}, time is fundamental to our
understanding of how events are ordered. It is clear that if an event occurs at
6:42, then it \emph{happens before} another event that occurs at 6:45.
Unfortunately, accurately measuring time in a distributed system is
infeasible~\cite{marzullo1984maintaining, sampath2012synchronization,
schmid2000orthogonal}. Clocks on different servers within a distributed system
drift apart, so servers cannot agree on a single global notion of time, and
thus they cannot agree on a single global \emph{total order} of events that
respects the real time ordering of events. However, as Lamport showed in
\cite{lamport1978time}, it \emph{is} possible for servers to agree on a global
\emph{partial order} of events that respects the global passage of time. This
partial ordering of events also dictates which events can causally affect each
other.

To make this partial ordering and notion of causality precise, we consider a
set of single-threaded servers that communicate over the network. Every server
$a$ serially executes a sequence of events $a_1, a_2, a_3, \ldots$ where each
event $a_i$ represents the action of server $a$ either (1) performing local
computation, (2) sending a message to another server, or (3) receiving a
message from another server.

% For example, consider the execution illustrated in \figref{Causality}. There
% are three servers ($a$, $b$, and $c$) and eight events ($a_1, a_2, \ldots,
% c_2$).  $a_1$ represents a local computation on $a$, $a_2$ represents server
% $a$ sending a message to server $b$, and event $b_2$ represents the receipt of
% this message by server $b$.

% {\begin{figure}[ht]
  \centering
  \begin{tikzpicture}[yscale=0.5]
    \tikzstyle{timeline}=[-latex, draw, flatdarkgray]
    \tikzstyle{serial}=[-latex, draw, black]
    \tikzstyle{netmsg}=[-latex, draw, black, snake=snake,
                        segment amplitude=0.7mm, segment length=3.7mm]

    \tikzstyle{event}=[shape=circle, inner sep=2pt]
    \tikzstyle{aevent}=[event, fill=flatred]
    \tikzstyle{bevent}=[event, fill=flatgreen]
    \tikzstyle{cevent}=[event, fill=flatblue]
    \tikzstyle{labelshift}=[shift={(0, -0.1)}]

    % Process labels.
    \node (a) at (0, 2) {Server $a$};
    \node (b) at (0, 1) {Server $b$};
    \node (c) at (0, 0) {Server $c$};

    % Time lines.
    \draw[timeline] (a) to +(4.25, 0);
    \draw[timeline] (b) to +(4.25, 0);
    \draw[timeline] (c) to +(4.25, 0);

    % Events.
    \node[aevent, label={[labelshift]$a_1$}] (a1) at ($(a.east) + (0.25, 0)$) {};
    \node[aevent, label={[labelshift]$a_2$}] (a2) at ($(a1)     + (0.75, 0)$) {};
    \node[aevent, label={[labelshift]$a_3$}] (a3) at ($(a2)     + (1.25, 0)$) {};
    \node[bevent, label={[labelshift]$b_1$}] (b1) at ($(b.east) + (0.25, 0)$) {};
    \node[bevent, label={[labelshift]$b_2$}] (b2) at ($(b1)     + (1.5, 0)$) {};
    \node[bevent, label={[labelshift]$b_3$}] (b3) at ($(b2)     + (0.75, 0)$) {};
    \node[cevent, label={[labelshift]$c_1$}] (c1) at ($(c.east) + (0.5, 0)$) {};
    \node[cevent, label={[labelshift]$c_2$}] (c2) at ($(c1)     + (2.75, 0)$) {};

    % Arrows.
    \draw[serial] (a1) edge (a2) (a2) edge (a3);
    \draw[serial] (b1) edge (b2) (b2) edge (b3);
    \draw[serial] (c1) edge (c2);
    \draw[netmsg] (a2) to (b2);
    \draw[netmsg] (b3) to (c2);
  \end{tikzpicture}
  \caption{TODO}\figlabel{Causality}
\end{figure}
}

The \defword{happens before} relation $\happensbefore$ on events is the
smallest transitive relation such that
%
(1)
  if $a_i, a_j$ are two events within the same process, 
  and $i < j$, then $a_i$ happens before $a_j$, and
(2)
  if $a_i$ and $b_j$ are the sending and receiving of a message respectively,
  then $a_i$ happens before $b_j$.
%
% For example, in \figref{Causality}, $b_1 \happensbefore b_2$, and $a_1
% \happensbefore c_2$.
The happens before relation is a partial order that formalizes our intuitive
notion of which events can causally affect each other. An event $a_i$ can only
be caused by an event $b_j$ that happens before it.  The set $\setst{b_j}{b_j
\happensbefore a_i}$ is called the \defword{causal history} of $a_i$.

\subsection{Data Provenance}
Given a relational database instance $I$, a relational algebra query $Q$, and a
tuple $t$ in the output of the query, it is natural to ask why $t$ appears in
the output. For example, consider the relational database instance given in
\figref{DataProvenance} that describes users and friends in a social media
application.  And, consider the relational algebra query
\[
  \footnotesize
  Q \defeq \pi_{\ttt{name}}(
             \sigma_{\ttt{friend1} = \ttt{ecodd}}(
                \ttt{Users} \join_{\ttt{username} = \ttt{friend2}} \ttt{Friends}
             )
           )
\]
that returns the name of all of Edgar Codd's friends. Evaluating $Q$ on $I$
produces the tuple $t = \text{(Michael Jordan)}$.

{\begin{figure}[ht]
  \centering

  \begin{minipage}{0.4\columnwidth}
    \ttt{Owners}

    \begin{tabular}{|l|l|}
      \hline
      \ttt{owner} & \ttt{pet} \\\hline
      Ava            & Mimi \\
      Ava            & Harry \\\hline
    \end{tabular}
  \end{minipage}%
  \begin{minipage}{0.4\columnwidth}
    \ttt{Pets}

    \begin{tabular}{|l|l|}
      \hline
      \ttt{pet} & \ttt{likes} \\\hline
      Mimi         & tuna \\
      Mimi         & sunlight \\
      Harry        & sunlight \\\hline
    \end{tabular}
  \end{minipage}
  \caption{An example database instance}\figlabel{DataProvenance}
\end{figure}
}

Intuitively, $t = \text{(Michael Jordan)}$ is present in the output $Q(I)$ for
two reasons: (1) the existence of the (ecodd, jumpman) and (jumpman, Michael
Jordan) tuples and (2) the existence of the (ecodd, mlpro) and (mlpro, Michael
Jordan) tuples.  \defword{\Whyprovenance{}}~\cite{buneman2001and,
cheney2009provenance} formalizes this intuition. The
\whyprovenance{}\footnote{
  For a formal definition of \whyprovenance{}, we refer the reader to
  \cite{cheney2009provenance}. For our purposes, an informal understanding of
  \whyprovenance{} is sufficent.
}
of a tuple $t$ with respect to query $Q$ and database instance $I$, denoted
$\Why{Q, I, t}$, is a set $J_1, \ldots, J_n$ of subinstances of $I$ where each
subinstance $J_i \subseteq I$ suffices to produce $t$ (i.e.\ $t \in Q(J_i)$).
These subinstances are called \defword{witnesses of $t$}, and a witness $J_i$
is called a \defword{minimal witness of $t$} if no proper subinstance of $J_i$
is also a witness of $t$. The \defword{minimal \whyprovenance{}} of $t$,
denoted $\MWhy{Q, I, t}$, is the set of the minimal witnesses in $\Why{Q, I,
t}$. It can be shown that $\MWhy{Q, I, t}$ is exactly the set of minimal
witnesses of $t$~\cite{cheney2009provenance}.

Returning to our example above, the \whyprovenance{} of the (Michael Jordan)
tuple is the set $\set{J_1, J_2}$ where
\begin{align*}
  J_1 &= \set{(\text{ecodd}, \text{jumpman}), (\text{jumpman}, \text{Michael Jordan})} \\
  J_2 &= \set{(\text{ecodd}, \text{mlpro}), (\text{mlpro}, \text{Michael Jordan})}
\end{align*}
$J_1$ and $J_2$ are minimal witnesses, so the \whyprovenance{} and minimal
\whyprovenance{} of $t = \text{(Michael Jordan)}$ are the same.

% The definition of \Whyprovenance{} for select ($\sigma$), project ($\pi$), join
% ($\join$), union ($\cup$) queries taken from \cite{cheney2009provenance} is as
% follows:
% \begin{align*}
%   \Why{\set{t}, I, u} &= \begin{cases}
%     \set{\emptyset}, & \text{if $(t = u)$,} \\
%     \emptyset, & \text{otherwise.} \\
%   \end{cases} \\
%   \Why{R, I, t} &= \begin{cases}
%     \set{\set{t}}, & \text{if $t \in R(I)$,} \\
%     \emptyset, & \text{otherwise.} \\
%   \end{cases} \\
%   \Why{\sigma_\theta(Q), I, t} &= \begin{cases}
%     \Why{Q, I, t}, & \text{if $\theta(t)$,} \\
%     \emptyset, & \text{otherwise.} \\
%   \end{cases} \\
%   \Why{\pi_U(Q), I, t} &= \bigcup_{\setst{u \in Q(I)}{t = u[U]}} \Why{Q, I, u} \\
%   \Why{Q_1 \join Q_2} &= \Why{Q_1, I, t} \Cup \Why{Q_2, I, t} \\
%   \Why{Q_1 \cup Q_2} &= \Why{Q_1, I, t} \cup \Why{Q_2, I, t} \\
% \end{align*}

\subsection{State Machines}
It is common to model servers---like key-value stores or relational
databases---as deterministic state machines that repeatedly receive requests,
update their state, and send replies~\cite{schneider1990implementing,
lamport1998part}. More precisely, a \defword{deterministic state machine} $M =
(S, s_0, \Sigma, \Lambda, \delta, \epsilon)$ consists of
  a (potentially infinite) set $S$ of states,
  a start state $s_0 \in S$,
  an input alphabet $\Sigma$,
  an output alphabet $\Lambda$,
  a transition function $\delta: S \times \Sigma \to S$, and
  an output function $\epsilon: S \times \Sigma \to \Lambda$.
A state machine $M$ begins in state $s_0$ and repeatedly receives inputs $a \in
\Sigma$. Upon receiving an input $a$, $M$ transitions from state $s$ to state
$\delta(s, a)$ and outputs $\epsilon(s, a)$.

In our work, we need to reason about specific sub-inputs to a state machine, in
the spirit of \whyprovenance{}, so we introduce some notation here. We refer to
an ordered sequence of inputs received by a state machine as a \defword{trace}
$T = a_1 a_2 \ldots a_n \in \Sigma^*$. A \defword{subtrace} $T'$ of $T$ is a
subsequence $T' = a_{i_1} a_{i_2} \ldots a_{i_m}$ where $i_1, i_2, \ldots, i_m$
are distinct elements of $1, 2, \ldots, n$ in ascending order. Note that a
subsequence does \emph{not} have to be contiguous. For example, $T' = a_1 a_3$
is a subtrace of $T = a_1 a_2 a_3 a_4$. If $T'$ is a subtrace of $T$, then a
\defword{supertrace of $T'$ in $T$} is a subtrace of $T$ that contains every
element of $T'$. If $T_1$ and $T_2$ are two traces, we write the concatenation
of $T_1$ and $T_2$ as $T_1T_2$. Similarly, if $T \in \Sigma^*$ is a trace, and
$a \in \Sigma$ is an input, we let $Ta$ denote the trace produced by appending
$a$ to the end of $T$. An illustrative example of the definition of subtraces
and supertraces is given in \figref{Traces}.

{\begin{center}
  \begin{tikzpicture}[yscale=5, xscale=6]
    \newcommand{\lw}{0pt}
    \newcommand{\rightof}[1]{($(#1.east) - (\lw, 0)$)}

    \tikzstyle{input}=[draw, minimum width=0.75in, minimum height=0.75in,
      anchor=west, inner sep=0pt, fill opacity=0.4,
      text opacity=1, line width=\lw]
    \tikzstyle{hiddeninput}=[input, dashed, dash pattern=on 0.5cm off 0.5cm]
    \tikzstyle{a1}=[input, fill=flatred]
    \tikzstyle{a2}=[input, fill=flatgreen]
    \tikzstyle{a3}=[input, fill=flatblue]
    \tikzstyle{a4}=[input, fill=flatpurple]

    \newcommand{\trace}[1]{
      \node[hiddeninput] (a1hidden) at (#1) {};
      \node[hiddeninput] (a2hidden) at \rightof{a1hidden} {};
      \node[hiddeninput] (a3hidden) at \rightof{a2hidden} {};
      \node[hiddeninput] (a4hidden) at \rightof{a3hidden} {};
    }
    \newcommand{\ai}{\node[a1] (a1) at (a1hidden.west) {$a_1$};}
    \newcommand{\aii}{\node[a2] (a2) at (a2hidden.west) {$a_2$};}
    \newcommand{\aiii}{\node[a3] (a3) at (a3hidden.west) {$a_3$};}
    \newcommand{\aiv}{\node[a4] (a4) at (a4hidden.west) {$a_4$};}
    \newcommand{\centertext}[1]{%
      \node[anchor=south] at (a2hidden.north east) {#1};
    }

    % Trace.
    \begin{scope}
      \trace{0, 0} \ai{} \aii{} \aiii{} \aiv{}
      % The phantom p's are here to vertically align all the text at the top of
      % this figure.
      \centertext{\phantom{p}trace $T$\phantom{p}};
    \end{scope}

    % Subtrace.
    \begin{scope}[shift={(0, -1)}]
      \trace{0, 0} \ai{} \aiii{}
      \centertext{subtrace $T'$};
    \end{scope}

    % Subtraces.
    \begin{scope}[shift={(2, 0)}]
      \trace{0, 0}
      % The phantom p's are here to vertically align all the text at the top of
      % this figure.
      \centertext{\phantom{p}subtraces of $T'$\phantom{p}};
      \trace{0, 0} \ai{} \aiii{}
      \trace{0, -0.5}  \aiii{}
      \trace{0, -1} \ai{}
      \trace{0, -1.5}
    \end{scope}

    % Supertraces.
    \begin{scope}[shift={(4, 0)}]
      \trace{0, 0}
      \centertext{supertraces of $T'$ in $T$};
      \trace{0, 0} \ai{} \aiii{}
      \trace{0, -0.5} \ai{} \aii{} \aiii{}
      \trace{0, -1} \ai{} \aiii{} \aiv{}
      \trace{0, -1.5} \ai{} \aii{} \aiii{} \aiv{}
    \end{scope}
  \end{tikzpicture}
\end{center}
}

$\delta$ takes in a state $s \in \Sigma$ and a single input $a \in \Sigma$.
It is convenient to extend $\delta$ to a function $\delta^*: S \times \Sigma^*
\to S$ that takes in a state $s$ and a trace $T \in \Sigma^*$ and outputs the
state reached after sequentially executing the inputs in $T$ starting in state
$s$. Similarly, we can extend $\epsilon$ to a function $\epsilon: S \times
\Sigma^+ \to \Lambda$ which takes in a state $s$ and a non-empty trace $T = a_1
\ldots a_n \in \Sigma^+$ and returns $\epsilon(\delta^*(s, a_1 a_2 \ldots
a_{n-1}), a_n)$: the final output produced from sequentially executing every
input in $T$ starting in state $s$.
